\documentclass[11pt, a4paper]{article}

\usepackage{amsmath}
\usepackage{amssymb}
\usepackage{array}
\usepackage{longtable}
\usepackage{multirow}
\usepackage{listings}
\usepackage{xcolor}
\usepackage{siunitx}
\usepackage{seqsplit}
\usepackage{kotex}
\usepackage{booktabs}
\usepackage{caption}
\usepackage{tabularx}

\usepackage{makecell}


\usepackage{fontspec}
\setmainfont{Noto Serif CJK KR}

\usepackage{graphicx}
\usepackage{geometry}
\geometry{margin=0.5in}

\usepackage{hyperref}
\hypersetup{
    colorlinks=true,
    linkcolor=blue,
    urlcolor=blue,
    citecolor=blue
}

\usepackage{float}

\title{MIMIC 기반 기존 연구 리뷰 및 재현 보고서 \\ Appendix}
\author{2024404060 강민혁}
\date{}

\begin{document}

\maketitle

\section{기흉이란}


\begin{figure}[htbp]
  \centering
  \includegraphics[width=1.0\textwidth]{figures/chest_pa_with_visualization.png}
  \caption{Pneumothorax X-ray with Visualization}
  \label{fig:chest_pa_with_visualization}
\end{figure}


폐에 여러 원인으로 구멍이 생겨 새어나온 공기가 흉강에 차서, 폐를 압박해 찌그러뜨리는 질환이며, 외상성 기흉은 외부의 물리적 충격으로 인해 폐나 흉벽이 손상되어 발생하는 기흉이다.


\section{소프트웨어 환경 및 버전}

\begin{table}[htbp]
  \centering
  \caption{Environment and Key Library Versions}
  \label{tab:environment}
  \renewcommand{\arraystretch}{1.1}
  \begin{tabular}{lc}
    \toprule
    \textbf{Software / Library} & \textbf{Version} \\
    \midrule
    Python & 3.12.12 \\
    CUDA & 13.0 \\
    \midrule
    Pandas & 2.3.3 \\
    NumPy & 2.3.4 \\
    SciPy & 1.16.3 \\
    Matplotlib & 3.10.8 \\
    Seaborn & 0.13.2 \\
    Statsmodels & 0.14.5 \\
    Scikit-learn & 1.7.2 \\
    XGBoost & 3.1.2 \\
    SHAP & 0.48.0 \\
    \bottomrule
  \end{tabular}
\end{table}

\end{document}

\section