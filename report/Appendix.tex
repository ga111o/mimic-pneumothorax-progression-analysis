\documentclass[11pt, a4paper]{article}

\usepackage{amsmath}
\usepackage{amssymb}
\usepackage{array}
\usepackage{longtable}
\usepackage{multirow}
\usepackage{listings}
\usepackage{xcolor}
\usepackage{siunitx}
\usepackage{seqsplit}
\usepackage{kotex}
\usepackage{booktabs}
\usepackage{caption}
\usepackage{tabularx}

\usepackage{makecell}


\usepackage{fontspec}
\setmainfont{Noto Serif CJK KR}

\usepackage{graphicx}
\usepackage{geometry}
\geometry{margin=0.5in}

\usepackage{hyperref}
\hypersetup{
    colorlinks=true,
    linkcolor=blue,
    urlcolor=blue,
    citecolor=blue
}

\usepackage{float}

\title{MIMIC 기반 기존 연구 리뷰 및 재현 보고서 \\ Appendix}
\author{2024404060 강민혁}
\date{}

\begin{document}

\maketitle

\section{기흉이란}


\begin{figure}[htbp]
  \centering
  \includegraphics[width=1.0\textwidth]{figures/chest_pa_with_visualization.png}
  \caption{Pneumothorax X-ray with Visualization}
  \label{fig:chest_pa_with_visualization}
\end{figure}


폐에 여러 원인으로 구멍이 생겨 새어나온 공기가 흉강에 차서, 폐를 압박해 찌그러뜨리는 질환이며, 외상성 기흉은 외부의 물리적 충격으로 인해 폐나 흉벽이 손상되어 발생하는 기흉이다.


\section{소프트웨어 환경 및 버전}

\begin{table}[htbp]
  \centering
  \caption{Environment and Key Library Versions}
  \label{tab:environment}
  \renewcommand{\arraystretch}{1.1}
  \begin{tabular}{lc}
    \toprule
    \textbf{Software / Library} & \textbf{Version} \\
    \midrule
    Python & 3.12.12 \\
    CUDA & 13.0 \\
    \midrule
    Pandas & 2.3.3 \\
    NumPy & 2.3.4 \\
    SciPy & 1.16.3 \\
    Matplotlib & 3.10.8 \\
    Seaborn & 0.13.2 \\
    Statsmodels & 0.14.5 \\
    Scikit-learn & 1.7.2 \\
    XGBoost & 3.1.2 \\
    SHAP & 0.48.0 \\
    \bottomrule
  \end{tabular}
\end{table}

\section{수식 정의}

\subsection{산소화 지수 (Oxygenation Index)}
주요 예측 인자인 Oxygenation Index의 산출 공식은 다음과 같다.
\begin{equation}
    \text{Oxygenation Index} = \frac{\text{FiO}_2 \times \text{MAP}}{\text{PaO}_2} \times 100
    \label{eq:oi}
\end{equation}

\subsection{복합 목적 함수 (Composite Objective Function)}
하이퍼파라미터 최적화를 위해 AUROC에 재현율 가중치를 결합한 목적 함수를 다음과 같이 구성하였다.
\begin{equation}
    \text{Score} = \text{AUROC} + \alpha \times \text{Recall}
    \label{eq:composite_objective}
\end{equation}

\subsection{제약 조건 기반 임계값 최적화}
모델의 최종 예측 단계에서 임상적 안전성을 보장하기 위한 제약 조건은 다음과 같다.
\begin{equation}
    \text{Recall}(t) \ge 0.45 \label{eq:recall_constraint}
\end{equation}
\begin{equation}
    \text{Accuracy}(t) \ge 0.85 \label{eq:accuracy_constraint}
\end{equation}

\section{예측 모델에 사용된 변수 정의}

\begin{table}[htbp]
  \centering
  \small
  \caption{Definitions of the 12 Key Variables Selected for the Predictive Model}
  \vspace{-0.5em}
  \label{tab:selected_variables}
  \renewcommand{\arraystretch}{1.2}
  \begin{tabularx}{\textwidth}{l l X}
    \toprule
    \textbf{Variable} & \textbf{Unit} & \textbf{Description} \\
    \midrule
    pH & - & Indicator of the acid-base status of the blood \\
    Hemoglobin & g/dL & Protein concentration reflecting the oxygen-carrying capacity of the blood \\
    PaO$_2$ & mmHg & Partial pressure of arterial oxygen; reflects dissolved oxygen tension \\
    Lactate & mmol/L & Marker of tissue hypoxia and anaerobic metabolism \\
    Oxygenation Index & - & Calculated index assessing the severity of respiratory failure \\
    SpO$_2$ & \% & Peripheral oxygen saturation; percentage of oxygen-bound hemoglobin \\
    Base Excess & mEq/L & Measure of metabolic acid-base status, independent of respiratory factors \\
    Heart Rate & bpm & Number of heartbeats per minute \\
    PaCO$_2$ & mmHg & Partial pressure of arterial carbon dioxide; indicator of alveolar ventilation \\
    Systolic BP & mmHg & Arterial blood pressure during ventricular contraction \\
    Diastolic BP & mmHg & Arterial blood pressure during ventricular relaxation \\
    Respiratory Rate & breaths/min & Number of breaths per minute \\
    \bottomrule
  \end{tabularx}
  \footnotesize
  \scriptsize
  \vspace{0.5em}
  \textit{Note: PaO$_2$, partial pressure of oxygen; SpO$_2$, peripheral oxygen saturation; PaCO$_2$, partial pressure of carbon dioxide; BP, blood pressure.}
\end{table}

\end{document}