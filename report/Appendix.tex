\documentclass[11pt, a4paper]{article}

\usepackage{amsmath}
\usepackage{amssymb}
\usepackage{array}
\usepackage{longtable}
\usepackage{multirow}
\usepackage{listings}
\usepackage{xcolor}
\usepackage{siunitx}
\usepackage{seqsplit}
\usepackage{kotex}
\usepackage{booktabs}
\usepackage{caption}
\usepackage{tabularx}

\usepackage{makecell}


\usepackage{fontspec}
\setmainfont{Noto Serif CJK KR}

\usepackage{graphicx}
\usepackage{geometry}
\geometry{margin=0.5in}

\usepackage{hyperref}
\hypersetup{
    colorlinks=true,
    linkcolor=blue,
    urlcolor=blue,
    citecolor=blue
}

\usepackage{float}

\title{MIMIC 기반 기존 연구 리뷰 및 재현 보고서 // Appendix}
\author{2024404060 강민혁}
\date{}

\begin{document}

\maketitle

\section{Pneumothorax}

d
\begin{figure}[htbp]
  \centering
  \includegraphics[width=1.0\textwidth]{figures/original_results.png}
  \caption{Original Research Performance Evaluation and SHAP Analysis Results. (Left) Model Performance Comparison Table for MIMIC-IV Internal Validation and EICU External Validation. (Right) SHAP Summary Plot for XGBoost Model, showing pH, Hgb, PO2 as the most influential factors in prediction.}
  \label{fig:original_results}
\end{figure}

\begin{table}[htbp]
  \centering
  \caption{Environment and Key Library Versions}
  \label{tab:environment}
  \renewcommand{\arraystretch}{1.1}
  \begin{tabular}{lc}
    \toprule
    \textbf{Software / Library} & \textbf{Version} \\
    \midrule
    Python & 3.12.12 \\
    CUDA & 13.0 \\
    \midrule
    XGBoost & 3.1.2 \\
    Scikit-learn & 1.7.2 \\
    SHAP & 0.48.0 \\
    Pandas & 2.3.3 \\
    NumPy & 2.3.4 \\
    SciPy & 1.16.3 \\
    \bottomrule
  \end{tabular}
\end{table}

\end{document}
