\documentclass[12pt, a4paper]{article}

\usepackage{amsmath}
\usepackage{amssymb}
\usepackage{booktabs}
\usepackage{array}
\usepackage{longtable}
\usepackage{multirow}
\usepackage{listings}
\usepackage{xcolor}
\usepackage{siunitx}
\usepackage{seqsplit}
\usepackage{kotex}

\usepackage{fontspec}
\setmainfont{Noto Serif CJK KR}

\usepackage{graphicx}
\usepackage{geometry}
\geometry{margin=1in}

\usepackage{hyperref}
\hypersetup{
    colorlinks=true,
    linkcolor=blue,
    urlcolor=blue,
    citecolor=blue
}

\usepackage{float}

\title{MIMIC-IV 기반 해석 가능한 머신러닝을 통한 외상성 중증 기흉 평가 논문 리뷰 및 재현}
\author{2024404060 \\ 강민혁}
\date{}

\begin{document}

\maketitle

\section{서론}
\subsection{선택한 논문의 기본 정보}

본 보고서에서 다루는 논문의 제목은 \textbf{``A New Evaluation Model for Traumatic Severe Pneumothorax Based on Interpretable Machine Learning''}이다~\cite{li2024}. 저자는 Jing Li, Yinzhen Lv, Jiayi Weng, Wei Chen, He Huang, Yuzhuo Zhao로 구성되어 있으며, 본 논문은 2024년 \textit{International Journal of Computers Communications \& Control}에 게재되었다.

위 논문은 XGBoost 등 머신러닝 기법들을 활용하여 MIMIC-IV 데이터셋을 기반으로 외상성 중증 기흉을 신속하고 해석 가능하게 평가하는 모델을 제안한 연구이다~\cite{li2024}.

\subsection{선택 이유 및 연구 주제의 중요성}
여기에 연구를 선택한 이유와 주제의 임상적/데이터 과학적 중요성을 기술한다.

\section{기존 논문 리뷰}
\subsection{연구 목적}
논문이 예측 혹은 분석하려 한 구체적인 목표를 요약한다.

\subsection{데이터 및 코호트 정의}
MIMIC 데이터에서 사용된 포함/제외 기준, 환자 수, 주요 변수 등을 정리한다.

\subsection{모델링 및 방법론}
사용된 알고리즘 및 모델, 피처 엔지니어링 전략, 학습 절차 등을 기술한다.

\subsection{평가 지표 및 주요 결과}
결과 지표를 정리하고, 논문의 주요 성능을 표 형태로 제시한다.

\begin{table}[H]
\centering
\caption{원 논문 주요 결과 요약}
\begin{tabular}{lcccc}
\toprule
모델 & AUROC & AUPRC & F1 & Accuracy \\
\midrule
논문 제시 모델 &  &  &  &  \\
비교 모델 1 &  &  &  &  \\
\bottomrule
\end{tabular}
\end{table}

\section{재현 방법}
\subsection{코호트 구성 및 데이터 파이프라인}
데이터 전처리, 필터링, 파이프라인 구성 단계를 설명한다.

\subsection{사용한 모델 및 하이퍼파라미터}
모델 구조, 학습률, 배치 크기 등 주요 하이퍼파라미터를 기술한다.

\subsection{실행 환경}
사용한 하드웨어 및 라이브러리 버전(예: Python, PyTorch, Scikit-learn, CUDA 등)을 명시한다.

\section{실험 결과}
\subsection{성능 지표 비교}
원 논문과 재현 실험 간의 결과를 비교한다.

\begin{table}[H]
\centering
\caption{성능 지표 비교 (원 논문 vs 재현 결과)}
\begin{tabular}{lcccc}
\toprule
모델 & AUROC & AUPRC & F1 & Accuracy \\
\midrule
원 논문 결과 &  &  &  &  \\
재현 결과 &  &  &  &  \\
\bottomrule
\end{tabular}
\end{table}

\subsection{시각화 결과}
ROC curve, PR curve, confusion matrix 등의 시각자료를 삽입한다.

% \begin{figure}[H]
% \centering
% \includegraphics[width=0.7\textwidth]{roc_curve.png}
% \caption{ROC Curve 비교 (원 논문 vs 재현 결과)}
% \end{figure}

% \begin{figure}[H]
% \centering
% \includegraphics[width=0.7\textwidth]{pr_curve.png}
% \caption{PR Curve 비교 (원 논문 vs 재현 결과)}
% \end{figure}

\section{논의 및 결론}
\subsection{재현 성공/실패 요인 분석}
결과가 일치하지 않았거나 차이가 발생한 원인을 데이터, 모델, 실행 환경 측면에서 분석한다.

\subsection{데이터/모델/환경 차이 검토}
원 논문과의 차이를 구체적으로 기술하고, 재현 결과에 미친 영향을 논의한다.

\subsection{연구에 대한 비판적 시각 및 개선 방안}
연구 설계 상의 한계점, 일반화 가능성, 향후 연구의 발전 방향을 제시한다.

\newpage
\begin{thebibliography}{9}

\bibitem{li2024}
Li, J., Lv, Y., Weng, J., Chen, W., Huang, H., \& Zhao, Y. (2024). 
A new evaluation model for traumatic severe pneumothorax based on interpretable machine learning. 
\textit{International Journal of Computers Communications \& Control, 19}(1), Article 6830. 
\url{https://doi.org/10.15837/ijccc.2025.1.6830}

\end{thebibliography}

\end{document}
